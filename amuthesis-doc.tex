%!TEX program=xelatex
\documentclass[oneside]{amuthesis}

\author{Bartłomiej Przybylski}
\album{123456}
\titlePL{Krótki podręcznik użytkownika klasy \texttt{amuthesis}}
\titleEN{A short user manual for \texttt{amuthesis} class}
\type{magisterska}
\field{matematyka}
\specialty{}
\supervisor{prof. UAM dr. hab. Jana Nowaka}
\date{Poznań, wrzesień 2017}

\usepackage{lipsum}

\begin{document}
\maketitle

\begin{streszczenie}
Klasa \texttt{amuthesis} została stworzona z myślą o studentach ostatnich lat studiów licencjackich, inżynierskich i magisterskich na Wydziale Matematyki i~Informatyki Uniwersytetu im. Adama Mickiewicza w Poznaniu, choć równie dobrze może być wykorzystywana na innych uczelniach. Dobrym zwyczajem jest bowiem składać prace dyplomowe z wykorzystaniem systemu \LaTeX{} i~bynajmniej nie dotyczy to wyłącznie prac z matematyki i informatyki. Niestety, samodzielnie przygotowanie wszystkich elementów składowych pracy dyplomowej wymaga nie tylko zaawansowanej znajomości systemu \LaTeX{} oraz zasad rządzących składem tekstu, ale przede wszystkim czasu. Klasa \texttt{amuthesis} dostarcza więc wszystko to, co jest potrzebne do stworzenia pięknej pracy dyplomowej w~języku polskim lub w~języku angielskim.
\end{streszczenie}

\begin{dedykacja}
Klasę \texttt{amuthesis} dedykuję wszystkim tym, dla których wygląd ma równie istotne znaczenie co wnętrze.
\end{dedykacja}

\tableofcontents

\chapter{Klasa \texttt{amuthesis}}

Główny plik klasy \texttt{amuthesis}, nazwany \texttt{amuthesis.cls}, zawiera definicje poleceń i otoczeń przydatnych w czasie tworzenia pracy dyplomowej. Opiera się przy tym na standardowych klasach: \texttt{book} dla prac w języku angielskim oraz \texttt{mwbk} dla prac w języku polskim. Klasa \texttt{amuthesis} współpracuje z najpopularniejszymi silnikami \LaTeX{}a:
\begin{itemize}
  \item \texttt{pdflatex},
  \item \texttt{xelatex},
  \item \texttt{lualatex}.
\end{itemize}

\section{Opcje}

Klasa \texttt{amuthesis} wspiera parametry opcjonalne, których przekazanie powoduje dostosowanie finalnego dokumentu do bieżacych potrzeb. Tabela~\ref{table:amuthesis-opcje} zawiera ich uproszczony opis.

\begin{table}[p]
  \caption{Opcje klasy \texttt{amuthesis}}
  \label{table:amuthesis-opcje}
  \centering
  \begin{tabular}{ccp{9cm}}
    \toprule
    Opcja & Domyślnie & Opis\\
    \midrule
    \texttt{polski} & Tak & Do pracy zostanie dołączony pakiet \texttt{polski}, a sam dokument zostanie oparty na klasie \texttt{mwbk}. W szczególności oznacza to, że wszystkie stosowane nazwy będą polskojęzyczne (np. ,,Rozdział''), a skład tekstu odbędzie się zgodnie z~polskimi normami.\\
    \midrule
    \texttt{english} & & Praca zostanie oparta na klasie \texttt{book}. Wszystkie stosowane nazwy będą anglojęzyczne (np. ,,Chapter''), a~skład tekstu odbędzie się zgodnie z~anglosaskimi normami.\\
    \midrule
    \texttt{logo} & & Zastąp logiem nazwę uczelni na początku strony tytułowej (plik \texttt{uam-logo.pdf}).\\
    \midrule
    \texttt{indent} & & Pierwszy akapit w ramach paragrafu zostanie wcięty. Jeśli do klasy nie zostanie przekazana opcja \texttt{indent}, to pierwsze akapity nie będą wcinane, niezależnie od języka, w którym składany jest dokument.\\
    \midrule
    \texttt{lineno} & & Wynikowy dokument zostanie wzbogacony o numerację wierszy.\\
    \midrule
    \texttt{oneside} & & Wygenerowany dokument będzie przygotowany do druku jednostronnego lub publikacji elektronicznej.\\
    \midrule
    \texttt{twoside} & Tak & Wygenerowany dokument będzie przygotowany do druku dwustronnego. Poszczególne części składowe pracy (np. rozdziały) będą się rozpoczynać zawsze od nieparzystej strony.\\
    \midrule
    \texttt{leftblank} & & Jeśli do klasy przekazano dodatkowo opcję \texttt{twoside}, to na pustej stronie przed kolejną częścią składową pracy (o ile taka występuje) zostanie umieszczony tekst ,,Ta strona jest pusta.'' (,,This page intentionally left blank.''). Tekst ten można zmienić korzystając z~polecenia \verb`\leftblank`.\\
    \midrule
    \texttt{swapthm} & & W stosowanych otoczeniach numerowanych, ich nazwa i numer zostaną zamienione miejscami (numer zostanie umieszczony przed nazwą).\\
    \bottomrule
  \end{tabular}
\end{table}

\section{Otoczenia}

Klasa \texttt{amuthesis} dostarcza szereg standardowych otoczeń, które mogą być wykorzystywane w tworzonym dokumencie. Tabela~\ref{table:amuthesis-otoczenia} zawiera ich pełną listę. Każde z takich otoczeń występuje pod jednym z dwóch oznaczeń, ale stosowana w dokumencie nazwa zależy wyłącznie od języka dokumentu. Elementy oddzielone w tabeli poziomą linią współdzielą numerację w ramach rozdziałów.

\begin{table}
  \caption{Otoczenia zdefiniowane w klasie \texttt{amuthesis}}
  \label{table:amuthesis-otoczenia}
  \centering
  \begin{tabular}{lll}
    \toprule
    Oznaczenia & Nazwa (pl) & Nazwa (en)\\
    \midrule
    \texttt{twierdzenie} / \texttt{theorem} & Twierdzenie & Theorem\\
    \texttt{dowod} / \texttt{proof} & Dowód & Proof\\
    \texttt{lemat} / \texttt{lemma} & Lemat & Lemma\\
    \texttt{hipoteza} / \texttt{statement} & Hipoteza & Statement\\
    \texttt{stwierdzenie} / \texttt{proposition} & Stwierdzenie & Proposition\\
    \texttt{wniosek} / \texttt{corollary} & Wniosek & Corollary\\
    \texttt{spostrzezenie} / \texttt{remark} & Spostrzeżenie & Remark\\
    \texttt{obserwacja} / \texttt{note} & Obserwacja & Note\\
    \midrule
    \texttt{definicja} / \texttt{definition} & Definicja & Definition\\
    \midrule
    \texttt{przyklad} / \texttt{example} & Przykład & Example\\
    \midrule
    \texttt{zadanie} / \texttt{task} & Zadanie & Task\\
    \texttt{cwiczenie} / \texttt{exercise} & Ćwiczenie & Exercise\\
    \bottomrule
  \end{tabular}
\end{table}

\section{Dodatkowe pakiety}

Jeśli dokument jest oparty na klasie \texttt{amuthesis}, to można w nim korzystać z~następujących pakietów bez dodatkowych działań:

\begin{verbatim}
fontspec  geometry  hyperref  xcolor    amsmath   amssymb
amsthm    graphicx  microtype booktabs  array     fancyhdr
tabularx  longtable makecell  verbatim  listings
\end{verbatim}

\chapter{Podstawowa struktura dokumentu}

Plik \texttt{thesis.tex} zawiera opis pustego dokumentu tworzonego w oparciu o klasę \texttt{amuthesis}. Możesz go uzupełnić treścią według własnego uznania.

\end{document}
