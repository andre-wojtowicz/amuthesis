%!TEX program=pdflatex
\documentclass[twoside,leftblank,polski]{amuthesis}

% Zdefiniuj kodowanie dokumentu
\usepackage[utf8]{inputenc}

% --- Autor pracy
\author{Bartłomiej Przybylski}
% --- Numer albumu
\album{123456}
% --- Tytuł pracy
\titlePL{Krótki podręcznik użytkownika klasy \texttt{amuthesis}}
\titleEN{A short user manual for \texttt{amuthesis} class}
% --- Typ pracy (inżynierska, licencjacka, magisterska)
\type{magisterska}
% --- Kierunek (w mianowniku)
\field{matematyka}
% --- Promotor (w dopełniaczu)
\supervisor{prof. UAM dr. hab. Adama Nowaka}
% --- Data złożenia pracy (Miasto, miesiąc rok)
\date{Poznań, wrzesień 2017}

% --- Płeć autora (M/K)
\stsex{M}
% --- Zgoda na udostępnienie pracy w czytelni (TAK/NIE)
\stread{TAK}
% --- Zgoda na udostępnienie pracy w zakresie ochrony (TAK/NIE)
\stprotect{TAK}

% --- Data podpisania oświadczenia (Miasto, data)
\stdate{Poznań, \today{} r.}

% Pozostałe opcje (odkomentuj pożądane)
%\leftblank{Ta strona jest pusta.}

% Umieść dodatkowe pakiety tutaj
\usepackage{lipsum}

\begin{document}

% Strona tytułowa
\maketitle

% Oświadczenie
\makestatement

% Blok abstraktu w języku polskim
\begin{streszczenie}
Klasa \texttt{amuthesis} została stworzona z myślą o studentach ostatnich lat studiów licencjackich, inżynierskich i magisterskich na Wydziale Matematyki i~Informatyki Uniwersytetu im. Adama Mickiewicza w Poznaniu, choć równie dobrze może być wykorzystywana na innych uczelniach. Dobrym zwyczajem jest bowiem składać prace dyplomowe z wykorzystaniem systemu \LaTeX{} i~bynajmniej nie dotyczy to wyłącznie prac z matematyki i informatyki. Niestety, samodzielnie przygotowanie wszystkich elementów składowych pracy dyplomowej wymaga nie tylko zaawansowanej znajomości systemu \LaTeX{} oraz zasad rządzących składem tekstu, ale przede wszystkim czasu. Klasa \texttt{amuthesis} dostarcza więc wszystko to, co jest potrzebne do stworzenia pięknej pracy dyplomowej w~języku polskim lub w~języku angielskim.
\end{streszczenie}

% Blok abstraktu w języku angielskim
\begin{abstract}
\lipsum[1]
\end{abstract}

% Opcjonalny blok dedykacji
\begin{dedykacja}
Klasę \texttt{amuthesis} dedykuję wszystkim tym, dla których wygląd ma równie istotne znaczenie co wnętrze.
\end{dedykacja}

% Spis treści
\tableofcontents

% Początek zasadniczej części dokumentu
\chapter{Klasa \texttt{amuthesis}}



\end{document}
