%!TEX program=pdflatex
\documentclass[oneside,polski]{amuthesis}

% Zdefiniuj kodowanie dokumentu
\usepackage[utf8]{inputenc}

% --- Autor pracy
\author{Bartłomiej Przybylski}
% --- Numer albumu
\album{123456}
% --- Tytuł pracy
\titlePL{Krótki podręcznik użytkownika klasy \texttt{amuthesis}}
\titleEN{A short user manual for \texttt{amuthesis} class}
% --- Typ pracy (inżynierska, licencjacka, magisterska)
\type{magisterska}
% --- Kierunek (w mianowniku)
\field{matematyka}
% --- Promotor (w dopełniaczu)
\supervisor{prof. UAM dr. hab. Adama Nowaka}
% --- Data złożenia pracy (Miasto, miesiąc rok)
\date{Poznań, wrzesień 2017}

% --- Płeć autora (M/K)
\stsex{M}
% --- Zgoda na udostępnienie pracy w czytelni (TAK/NIE)
\stread{TAK}
% --- Zgoda na udostępnienie pracy w zakresie ochrony (TAK/NIE)
\stprotect{TAK}

% --- Data podpisania oświadczenia (Miasto, data)
\stdate{Poznań, \today{} r.}

% Pozostałe opcje (odkomentuj pożądane)
%\leftblank{Ta strona jest pusta.}

% Umieść dodatkowe pakiety tutaj
\usepackage{lipsum}

\begin{document}

% Strona tytułowa
\maketitle

% Oświadczenie
\makestatement

% Blok abstraktu w języku polskim
\begin{streszczenie}
Klasa \texttt{amuthesis} została stworzona z myślą o studentach ostatnich lat studiów licencjackich, inżynierskich i magisterskich na Wydziale Matematyki i~Informatyki Uniwersytetu im. Adama Mickiewicza w Poznaniu, choć równie dobrze może być wykorzystywana na innych uczelniach. Dobrym zwyczajem jest bowiem składać prace dyplomowe z wykorzystaniem systemu \LaTeX{} i~bynajmniej nie dotyczy to wyłącznie prac z matematyki i informatyki. Niestety, samodzielnie przygotowanie wszystkich elementów składowych pracy dyplomowej wymaga nie tylko zaawansowanej znajomości systemu \LaTeX{} oraz zasad rządzących składem tekstu, ale przede wszystkim czasu. Klasa \texttt{amuthesis} dostarcza więc wszystko to, co jest potrzebne do stworzenia pięknej pracy dyplomowej w~języku polskim lub w~języku angielskim.
\end{streszczenie}

% Blok abstraktu w języku angielskim
\begin{abstract}
\lipsum[1]
\end{abstract}

% Opcjonalny blok dedykacji
\begin{dedykacja}
Klasę \texttt{amuthesis} dedykuję wszystkim tym, dla których wygląd ma równie istotne znaczenie co wnętrze.
\end{dedykacja}

% Spis treści
\tableofcontents

% Początek zasadniczej części dokumentu
\chapter{Klasa \texttt{amuthesis}}

Główny plik klasy \texttt{amuthesis}, nazwany \texttt{amuthesis.cls}, zawiera definicje poleceń i otoczeń przydatnych w czasie tworzenia pracy dyplomowej. Opiera się przy tym na standardowych klasach: \texttt{book} dla prac w języku angielskim oraz \texttt{mwbk} dla prac w języku polskim.

\section{Opcje}

Klasa \texttt{amuthesis} wspiera parametry opcjonalne, których przekazanie powoduje dostosowanie finalnego dokumentu do bieżacych potrzeb. Tabela~\ref{table:amuthesis-opcje} zawiera ich uproszczony opis.

\begin{table}[p]
  \caption{Opcje klasy \texttt{amuthesis}}
  \label{table:amuthesis-opcje}
  \begin{center}
  \begin{tabular}{ccp{9cm}}
    \toprule
    Opcja & Domyślnie & Opis\\
    \midrule
    \texttt{polski} & Tak & Do pracy zostanie dołączony pakiet \texttt{polski}, a sam dokument zostanie oparty na klasie \texttt{mwbk}. W szczególności oznacza to, że wszystkie stosowane nazwy będą polskojęzyczne (np. ,,Rozdział''), a skład tekstu odbędzie się zgodnie z~polskimi normami.\\
    \midrule
    \texttt{english} & & Praca zostanie oparta na klasie \texttt{book}. Wszystkie stosowane nazwy będą anglojęzyczne (np. ,,Chapter''), a skład tekstu odbędzie się zgodnie z~anglosaskimi normami.\\
    \midrule
    \texttt{indent} & & Pierwszy akapit w ramach paragrafu zostanie wcięty. Jeśli do klasy nie zostanie przekazana opcja \texttt{indent}, to pierwsze akapity nie będą wcinane, niezależnie od języka, w którym składany jest dokument.\\
    \midrule
    \texttt{lineno} & & Wynikowy dokument zostanie wzbogacony o numerację wierszy.\\
    \midrule
    \texttt{oneside} & Tak & Wygenerowany dokument będzie przygotowany do druku jednostronnego lub publikacji elektronicznej.\\
    \midrule
    \texttt{twoside} & & Wygenerowany dokument będzie przygotowany do druku dwustronnego. Poszczególne części składowe pracy (np. rozdziały) będą się rozpoczynać zawsze od nieparzystej strony.\\
    \midrule
    \texttt{leftblank} & & Jeśli do klasy przekazano dodatkowo opcję \texttt{twoside}, to na pustej stronie przed kolejną częścią składową pracy (o ile taka występuje) zostanie umieszczony tekst ,,Ta strona jest pusta.''. Tekst ten można zmienić korzystając z~polecenia \verb`\leftblank`.\\
    \bottomrule
  \end{tabular}
  \end{center}
\end{table}

\section{Otoczenia}

Klasa \texttt{amuthesis} dostarcza szereg standardowych otoczeń, które mogą być wykorzystywane w tworzonym dokumencie. Każde z takich otoczeń występuje pod jednym z dwóch oznaczeń, ale stosowana w dokumencie nazwa zależy wyłącznie od języka dokumentu. Tabela~\ref{table:amuthesis-otoczenia} zawiera ich pełną listę. Elementy oddzielone poziomą linię współdzielą numerację w ramach rozdziałów.

\begin{table}[p]
  \caption{Otoczenia zdefiniowane w klasie \texttt{amuthesis}}
  \label{table:amuthesis-otoczenia}
  \begin{center}
  \begin{tabular}{lll}
    \toprule
    Oznaczenia & Nazwa (pl) & Nazwa (en)\\
    \midrule
    \texttt{twierdzenie} / \texttt{theorem} & Twierdzenie & Theorem\\
    \texttt{dowod} / \texttt{proof} & Dowód & Proof\\
    \texttt{lemat} / \texttt{lemma} & Lemat & Lemma\\
    \texttt{hipoteza} / \texttt{statement} & Hipoteza & Statement\\
    \texttt{stwierdzenie} / \texttt{proposition} & Stwierdzenie & Proposition\\
    \texttt{wniosek} / \texttt{corollary} & Wniosek & Corollary\\
    \texttt{spostrzeżenie} / \texttt{remark} & Spostrzeżenie & Remark\\
    \texttt{obserwacja} / \texttt{note} & Obserwacja & Note\\
    \midrule
    \texttt{definicja} / \texttt{definition} & Definicja & Definition\\
    \midrule
    \texttt{przyklad} / \texttt{example} & Przykład & Example\\
    \midrule
    \texttt{zadanie} / \texttt{task} & Zadanie & Task\\
    \texttt{cwiczenie} / \texttt{exercise} & Ćwiczenie & Exercise\\
    \bottomrule
  \end{tabular}
  \end{center}
\end{table}

\begin{theorem}Lorem ipsum.\end{theorem}
\begin{proof}Lorem ipsum.\end{proof}
\begin{lemma}Lorem ipsum.\end{lemma}
\begin{statement}Lorem ipsum.\end{statement}
\begin{proposition}Lorem ipsum.\end{proposition}
\begin{corollary}Lorem ipsum.\end{corollary}
\begin{remark}Lorem ipsum.\end{remark}
\begin{note}Lorem ipsum.\end{note}
\begin{definition}Lorem ipsum.\end{definition}
\begin{example}Lorem ipsum.\end{example}
\begin{task}Lorem ipsum.\end{task}
\begin{exercise}Lorem ipsum.\end{exercise}

\section{Dodatkowe pakiety}

Jeśli dokument jest oparty na klasie \texttt{amuthesis}, to można w nim korzystać z~następujących pakietów bez dodatkowych działań:
\texttt{fontspec},  \texttt{geometry},  \texttt{hyperref},  \texttt{xcolor},
\texttt{amsmath},   \texttt{amssymb},   \texttt{amsthm},    \texttt{graphicx},
\texttt{microtype}, \texttt{booktabs},  \texttt{fancyhdr},  \texttt{array},
\texttt{tabularx},  \texttt{longtable}, \texttt{makecell}.

\end{document}
