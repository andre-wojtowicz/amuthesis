\documentclass[polski]{amuthesis}
% Opcje:
% --- polski (domyślnie)
% --- english
% --- lineno (włącz numerowanie wierszy)
% --- indent (wcinaj pierwszy akapit paragrafu)

% --- Autor pracy
\author{Bartłomiej Przybylski}
% --- Numer albumu
\album{123456}
% --- Tytuł pracy
\titlePL{Przykład zastosowania klasy \texttt{amuthesis}}
\titleEN{An example use of \texttt{amuthesis} class}
% --- Typ pracy (inżynierska, licencjacka, magisterska)
\type{magisterska}
% --- Kierunek (w mianowniku)
\field{matematyka}
% --- Promotor (w dopełniaczu)
\supervisor{prof. UAM dr. hab. Adama Nowaka}
% --- Data złożenia pracy (Miasto, miesiąc rok)
\date{Poznań, wrzesień 2017}

% --- Płeć autora (wybierz jedną)
\stmale
%\stfemale

% --- Zgoda na udostępnienie pracy w czytelni (TAK/NIE)
\stread{TAK}
% --- Zgoda na udostępnienie pracy w zakresie ochrony (TAK/NIE)
\stprotect{TAK}

% --- Data podpisania oświadczenia (Miasto, data)
\stdate{Poznań, \today{} r.}


\usepackage{lipsum}

\begin{document}

% Strona tytułowa
\maketitle

% Oświadczenie
\makestatement

% Opcjonalny blok abstraktu w języku polskim
\begin{streszczenie}
\lipsum[1]
\end{streszczenie}

% Opcjonalny blok abstraktu w języku angielskim
\begin{abstract}
\lipsum[1]
\end{abstract}

% Opcjonalny blok podziękowań
\begin{podziekowania}
\lipsum[5]
\end{podziekowania}

% Opcjonalny blok dedykacji
\begin{dedykacja}
\emph{Moja dedykacja}
\end{dedykacja}

% Spis treści
\tableofcontents

% Początek zasadniczej części dokumentu
\chapter{Test}

\lipsum[1]

\section{To jest paragraf}

\lipsum[2-3]

\subsection{DEF}

\lipsum[4]

\end{document}
