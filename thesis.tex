\documentclass[oneside,polski,logo]{amuthesis}

% Zdefiniuj kodowanie dokumentu
\usepackage[utf8]{inputenc}

% --- Autor pracy
\author{Jan Kowalski}
% --- Numer albumu
\album{123456}
% --- Tytuł pracy
\titlePL{Moja praca}
\titleEN{My thesis}
% --- Typ pracy (inżynierska, licencjacka, magisterska)
\type{magisterska}
% --- Kierunek (w mianowniku)
\field{matematyka}
% --- Promotor (w dopełniaczu)
\supervisor{prof. UAM dr. hab. Jana Nowaka}
% --- Data złożenia pracy (Miasto, miesiąc rok)
\date{Poznań, wrzesień 2017}

% --- Płeć autora (M/K)
\stsex{M}
% --- Zgoda na udostępnienie pracy w czytelni (TAK/NIE)
\stread{TAK}
% --- Zgoda na udostępnienie pracy w zakresie ochrony (TAK/NIE)
\stprotect{TAK}

% --- Data podpisania oświadczenia (Miasto, data)
\stdate{Poznań, \today{} r.}

% Pozostałe opcje (odkomentuj pożądane)
% \leftblank{Ta strona jest pusta.}

% Umieść dodatkowe pakiety tutaj
\usepackage{lipsum}

% Początek dokumentu
\begin{document}

% Strona tytułowa
\maketitle
% Oświadczenie
\makestatement

% Blok abstraktu w języku polskim
\begin{streszczenie}
\lipsum[1]
\end{streszczenie}

% Blok abstraktu w języku angielskim
\begin{abstract}
\lipsum[2]
\end{abstract}

% Opcjonalny blok dedykacji
\begin{dedykacja}
Tu możesz umieścić swoją dedykację.
\end{dedykacja}

% Spis treści
\tableofcontents

% Początek zasadniczej części dokumentu
\chapter{Treść mojej pracy}

\lipsum[3]

\end{document}
